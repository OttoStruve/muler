\documentclass[modern]{aastex631}
\bibliographystyle{aasjournal}

% \usepackage{fontspec}
% \usepackage[T1]{fontenc}
% \usepackage{newtxsf}
% \setmainfont{Fira Sans Book}[Scale=1.0]

\usepackage[caption=false]{subfig}
\usepackage{booktabs}

\begin{document}
\shorttitle{Echelle Spectroscopy API}
\shortauthors{Gully-Santiago et al. }
\title{An Application Programming Interface for Echelle Spectroscopy}

\author{Michael Gully-Santiago}
\affiliation{University of Texas at Austin Department of Astronomy}

\author{TBD}
\affiliation{TBD}


\begin{abstract}

Modern \'echelle spectrographs produce information-rich echellograms that undergo standard reduction procedures to produce extracted 1D spectra, typically chunked into echelle orders.  The final post-processing steps for echelle orders are often left to the end-user scientists, since the order of opertations and algorithm choice may depend on the scientific application.  These steps---while uncontroversial and routine---act as a barrier to entry to newcomers, and act as a tax on scientific innovation since teams have to re-invent the wheel to overcome implementation complexity, before getting to the already inherently complex scientific enterprise.  Here we assemble a collection of standard post-processing algorithms into a single easy-to-use Application Programming Interface (API).  The open source permissively licensed Python 3 implementation \texttt{muler} lowers the barrier to entry and accelerates the investigation of astronomical spectra.  The framework currently supports data from the HPF, Keck NIRSPEC, and IGRINS spectrographs, and is extensible to others.

\end{abstract}

\keywords{High resolution spectroscopy (2096)}

\section{Introduction}\label{sec:intro}

Here is an annotated bibliography.

\begin{deluxetable}{chc}
  \tablecaption{Annotated bibliography for intro\label{table1}}
  \tablehead{
  \colhead{Reference} & \nocolhead{two} & \colhead{Key idea}
  }
  \startdata
  \citet{czekala15} & - & \texttt{Starfish} \\
  \citet{gullysantiago17} & - & Starfish modified for mixture models\\
  \citet{2019AJ....158..164B} & - & \texttt{wobble} \\
  Barentsen et al. & - & \texttt{lightkurve} \\
  \enddata
\end{deluxetable}

\newpage

\begin{acknowledgements}

The authors acknowledge the Texas Advanced Computing Center (TACC, \url{http://www.tacc.utexas.edu}) at The University of Texas at Austin for providing HPC resources that have contributed to the research results reported within this paper.
\end{acknowledgements}

\clearpage


\facilities{HET (HPF), IGRINS (Gemini South), NIRSPEC (Keck)}

\software{  pandas \citep{mckinney10, reback2020pandas},
  emcee \citep{foreman13},
  matplotlib \citep{hunter07},
  astroplan \citep{astroplan2018},
  astropy \citep{exoplanet:astropy13,exoplanet:astropy18},
  exoplanet \citep{exoplanet:exoplanet},
  numpy \citep{harris2020array},
  scipy \citep{jones01},
  ipython \citep{perez07},
  starfish \citep{czekala15},
  bokeh \citep{bokehcite},
  seaborn \citep{waskom14},
  pytorch \citep{NEURIPS2019_9015}} % No pytorch yet!


\bibliography{ms}

\end{document}
